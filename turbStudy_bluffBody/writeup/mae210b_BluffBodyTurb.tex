\documentclass[journal]{new-aiaa}
% \documentclass[conf]{new-aiaa} %for conference papers
\usepackage[utf8]{inputenc}

\usepackage{graphicx}
\usepackage{amsmath}
\usepackage[version=4]{mhchem}
\usepackage{siunitx}
\usepackage{longtable,tabularx}
\setlength\LTleft{0pt}

% \usepackage{amsmath}          % for formula writing (i.e. 'split', etc)
\usepackage{rotate}           %rotate/mirror images
\usepackage{cancel}           %draw lines through math to show "goes to zero"
\usepackage{xfrac}            %allows slated and side fractions
\usepackage{subcaption}       %allows captioning individual subfigures
\usepackage[mode=buildnew]{standalone}% requires -shell-escape
  % compile with `pdflatex -shell-escape main` or `xelatex  -shell-escape main`

\usepackage{float} %Force figures to exact location in doc (use '[H]' option)

\usepackage{tikz}             %for creating vector graphics diagrams
\usetikzlibrary{backgrounds}  %put backgrounds behind tikz figures
\usetikzlibrary{calc}         %perform calculations within $$
\usetikzlibrary{positioning}  %position tikz elements using "right of, etc"
\usetikzlibrary{angles}       %label angles between lines with arcs
\usetikzlibrary{quotes}       %Put angle label in quotes
\usetikzlibrary{patterns}     %Patterns to fill shapes with




\title{Review of Analysis and Modeling Techniques for Incompressible, Turbulent Bluff-Body Wakes}

\author{Logan D. Halstrom\footnote{Graduate Student, Mechanical And Aerospace Engineering Department, One Shields Avenue} and Federico Zabaleta\footnote{Graduate Student, Civil and Environmental Engineering Department, One Shields Avenue}}
\affil{University of California, Davis, California, 95616}



%%%%%%%%%%%%%%%%%%%%%%%%%%%%%%%%%%%%%%%%%%%%%%%%%%%%%%%%%%%%%%%%%%%%%%%%
\begin{document}

\maketitle

%%%%%%%%%%%%%%%%%%%%%%%%%%%%%%%%%%%%%%%%%%%%%%%%%%%%%%%%%%%%%%%%%%%%%%%%
\begin{abstract} %%%%%%%%%%%%%%%%%%%%%%%%%%%%%%%%%%%%%%%%%%%%%%%%%%%%%%%
%%%%%%%%%%%%%%%%%%%%%%%%%%%%%%%%%%%%%%%%%%%%%%%%%%%%%%%%%%%%%%%%%%%%%%%%

\textcolor{red}{abstract here}
\textcolor{red}{\emph{LH\&FZ}}

\end{abstract}



%%%%%%%%%%%%%%%%%%%%%%%%%%%%%%%%%%%%%%%%%%%%%%%%%%%%%%%%%%%%%%%%%%%%%%%%
\section*{Nomenclature} %%%%%%%%%%%%%%%%%%%%%%%%%%%%%%%%%%%%%%%%%%%%%%%%
%%%%%%%%%%%%%%%%%%%%%%%%%%%%%%%%%%%%%%%%%%%%%%%%%%%%%%%%%%%%%%%%%%%%%%%%

\textcolor{red}{\emph{LH\&FZ}}

{\renewcommand\arraystretch{1.0}
\noindent\begin{longtable*}{@{}l @{\quad=\quad} l@{}}
$\rho$ & density, $kg/m^3$\\
\multicolumn{2}{@{}l}{Subscripts}\\
$()_{\infty}$ & freestream quantity\\
\multicolumn{2}{@{}l}{Acronyms}\\
CFD & Computational Fluid Dynamics\\
\end{longtable*}}



%%%%%%%%%%%%%%%%%%%%%%%%%%%%%%%%%%%%%%%%%%%%%%%%%%%%%%%%%%%%%%%%%%%%%%%%
\section{Introduction} \label{sec:intro}
%%%%%%%%%%%%%%%%%%%%%%%%%%%%%%%%%%%%%%%%%%%%%%%%%%%%%%%%%%%%%%%%%%%%%%%%




\lettrine{I}{ntro} sentence to paper should have this fancy capitalization.

\begin{itemize}
    \item Driving Physical Phenomena \textcolor{red}{\emph{FZ}}

    \begin{itemize}
        \item differences from potential flow
        \item blunt/bluff body definition, differences from streamlined body flow
        \item massively separated flow
        \item base pressure
        \item wake
    \end{itemize}
    \item Real World Applications \textcolor{red}{\emph{LH}}
    \begin{itemize}
        \item parachute
        \item reentry capsule
        \item vehicles
        \item buildings
        \item show similarity between cylinder/sphere wake and more complex bluff body
    \end{itemize}
\end{itemize}






\emph{Big whorls have little whorls, which feed on their velocity, and little whorls have lesser whorls, and so on to viscosity (in the molecular sense).}
Richardson (1922) \cite{richardson1922weather}




%%%%%%%%%%%%%%%%%%%%%%%%%%%%%%%%%%%%%%%%%%%%%%%%%%%%%%%%%%%%%%%%%%%%%%%%
\section{Experimental Methods And Results} \label{sec:experimentalmethods}
%%%%%%%%%%%%%%%%%%%%%%%%%%%%%%%%%%%%%%%%%%%%%%%%%%%%%%%%%%%%%%%%%%%%%%%%

\textcolor{red}{\emph{FZ}}

\begin{itemize}
    \item Historical Study
    \item Experimental techniques
    \begin{itemize}
        \item ballistic range?
    \end{itemize}
    \item Applications
    \begin{itemize}
        \item Simple cases: cylinder/sphere
        \begin{itemize}
            \item Drag vs Re?
            \item Wake velocity profiles?
            \item Wake structure?
        \end{itemize}
        \item Sharp vs bluff: sphere vs cube
        \item Complex cases: capsule/building
    \end{itemize}
\end{itemize}








%%%%%%%%%%%%%%%%%%%%%%%%%%%%%%%%%%%%%%%%%%%%%%%%%%%%%%%%%%%%%%%%%%%%%%%%
\section{Computational Methods and Results} \label{sec:computationalmethods}
%%%%%%%%%%%%%%%%%%%%%%%%%%%%%%%%%%%%%%%%%%%%%%%%%%%%%%%%%%%%%%%%%%%%%%%%

\textcolor{red}{\emph{LH}}

\begin{itemize}
    \item Historical Study
    \item Computational techniques
    \item Applications
    \begin{itemize}
        \item Simple cases: cylinder/sphere
        \item Sharp vs bluff: sphere vs cube
        \item Complex cases: capsule/building
    \end{itemize}
\end{itemize}

%%%%%%%%%%%%%%%%%%%%%%%%%%%%%%%%%%%%%%%%%%%%%%%%%%%%%%%%%%%%%%%%%%%%%%%%
\subsection{Turbulence Modeling Aspects} \label{subsec:turbulencemodeling}

\textcolor{red}{\emph{LH}}

\begin{itemize}
    \item Compare turbulence model performance for sphere/cylinder
    \begin{itemize}
        \item SA
        \item SST
        \item SAS
        \item URANS
        \item LES
        \item DES
        \item DNS?
    \end{itemize}
\end{itemize}


LIST ALL FIGURES HEAR, REORDER AND DESCRIBE LATER


%%% SAS CYLINDER
%%\vspace{-2em}
% \begin{figure}[htb]
\begin{figure}[H]
\begin{center}
\includegraphics[width=0.5\textwidth]{Images/logan/menter2005scaleadaptive_cylinderwake.pdf}
\caption{ SAS vs URANS cylinder \cite{menter2005scaleadaptive} }
\label{fig:sasvsuranscylinder}
\end{center}
\end{figure}
%%\vspace{-2em}

Figure 5: Resolved structures for cylinder in crossflow using different constants FSAS




%%% LES VS RANS CYLINDER
%%\vspace{-2em}
% \begin{figure}[htb]
\begin{figure}[H]
\begin{center}
\includegraphics[width=0.5\textwidth]{Images/logan/catalano_2003numerical_UnsteadyURANSvsLES.pdf}
\caption{ cylinder les vs urans instantaneous \cite{catalano2003numerical} }
\label{fig:lesvsuranscylinderinstant}
\end{center}
\end{figure}
%%\vspace{-2em}

Instantaneous vorticity magnitude at a given spanwise cut for flow over a circular cylinder at ReD 1⁄4 1   106. 25 contour levels from xD=U1 1⁄4 1 to xD=U1 1⁄4 575 (exponential distribution) are plotted.


%%% LES VS RANS CYLINDER
%%\vspace{-2em}
% \begin{figure}[htb]
\begin{figure}[H]
\begin{center}
\includegraphics[width=0.5\textwidth]{Images/logan/catalano_2003numerical_SteadyURANSvsLES.pdf}
\caption{ cylinder les vs urans averaged \cite{catalano2003numerical} }
\label{fig:lesvsuranscylinderaveraged}
\end{center}
\end{figure}
%%\vspace{-2em}

Fig. 5. Mean streamwise velocity distribution predicted by LES and URANS. 45 contour levels from U=U1 1⁄4  0:2 to U=U1 1⁄4 1:7 are plotted.

%%% LES VS RANS CYLINDER
%%\vspace{-2em}
% \begin{figure}[htb]
\begin{figure}[H]
\begin{center}
\includegraphics[width=0.5\textwidth]{Images/logan/catalano_2003numerical_VelocityProfiles.pdf}
\caption{ cylinder les vs urans velocity profiles \cite{catalano2003numerical} }
\label{fig:lesvsuranscylindervelprofile}
\end{center}
\end{figure}
%%\vspace{-2em}

Fig. 6. Mean streamwise and vertical velocities at x=D 1⁄4 0:75 (upper figures) and x=D 1⁄4 1:50 (lower figures): (—) LES; (– –) URANS



%%%%%%%%%%%%%%%%%%%%%%%%%%%%%%%%%%%%%%%%%%%%%%%%%%%%%%%%%%%%

%%% LES VS RANS CYLINDER GRID
%%\vspace{-2em}
% \begin{figure}[htb]
\begin{figure}[H]
\begin{center}
\includegraphics[width=0.5\textwidth]{Images/logan/travin2000detachededdy_grid.pdf}
\caption{ cylinder les vs rans grid \cite{travin2000detachededdy} }
\label{fig:lesvsranscylindergrid}
\end{center}
\end{figure}
%%\vspace{-2em}

Figure1. Medium computational grid, CaseTS2.Innerblock150×36,wakeblock74×36, outer block 59 × 30. The three blocks meet near x = 1.06, y = 1.03.  Grid for spalart cylinders.


%%% LES VS RANS CYLINDER
%%\vspace{-2em}
% \begin{figure}[htb]
\begin{figure}[H]
\begin{center}
\includegraphics[width=0.5\textwidth]{Images/logan/spalart2000strategies_CylinderLESvsRANS.pdf}
\caption{ cylinder les vs rans \cite{spalart2000strategies} }
\label{fig:lesvsranscylinder}
\end{center}
\end{figure}
%%\vspace{-2em}


grid for LES shown above (actual simulations were DES)




%%% CYLINDER WITH VARIOUS TYPES OF NUMERICAL MODELING
%%\vspace{-2em}
% \begin{figure}[htb]
\begin{figure}[H]
\begin{center}
\includegraphics[width=0.3\textwidth]{Images/logan/spalart2009detachededdy_CylinderVariousTurbModels.jpeg}
\caption{ cylinder simulation with RANS, 2DURANS, 3DURANS, SSTDES, SADES \cite{spalart2009detachededdy} }
\label{fig:cylinderturbmodels}
\end{center}
\end{figure}
%%\vspace{-2em}

Vorticity isosurfaces by a circular cylinder: ReD = 5 × 104, laminar separation. Experimental drag
coefficient Cd = 1.15–1.25. (a) Shear-stress transport (SST) turbulence model steady Reynolds-averaged
Navier-Stokes (RANS), Cd = 0.78; (b) SST 2D unsteady RANS, Cd = 1.73; (c) SST 3D unsteady RANS,
Cd = 1.24; (d ) Spalart-Allmaras (SA) detached-eddy simulation (DES), coarse grid, Cd = 1.16; (e) SA DES,
fine grid, Cd = 1.26; ( f ) SST DES, fine grid, Cd = 1.28. Figure courtesy of A. Travin



illustrates the response of DES to grid refinement in its LES region.

DES solutions with different base RANS models are not sensitive to
model choice in the LES region (as opposed to the RANS region, particularly if separation occurs).





%%%%%%%%%%%%%%%%%%%%%%%%%%%%%%%%%%%%%%%%%%%%%%%%%%%%%%%%%%%%

%%% F15 DES
%%\vspace{-2em}
% \begin{figure}[htb]
\begin{figure}[H]
\begin{center}
\includegraphics[width=0.45\textwidth]{Images/logan/forsythe2004detachededdy_f15grid.pdf}
\includegraphics[width=0.45\textwidth]{Images/logan/spalart2009detachededdy_f15des.pdf}
\caption{ F-15 DES grid (left) \cite{forsythe2004detachededdy} vorticity isocontours (right) \cite{spalart2009detachededdy} }
\label{fig:f15des}
\end{center}
\end{figure}
%%\vspace{-2em}



%%% CAR DES
%%\vspace{-2em}
% \begin{figure}[htb]
\begin{figure}[H]
\begin{center}
\includegraphics[width=0.5\textwidth]{Images/logan/spalart2009detachededdy_carDES.pdf}
\caption{ car DES isocontours \cite{mendonca2002towards} }
\label{fig:cardes}
\end{center}
\end{figure}
%%\vspace{-2em}









%%%%%%%%%%%%%%%%%%%%%%%%%%%%%%%%%%%%%%%%%%%%%%%%%%%%%%%%%%%%


%%% CURVE BACKSTEP VELOCITY PROFILE LES VS RANS
%%\vspace{-2em}
% \begin{figure}[htb]
\begin{figure}[H]
\begin{center}
\includegraphics[width=0.5\textwidth]{Images/logan/durbin2018some_BackstepLESvsRANS.pdf}
\caption{curve backstep velocity profile les vs rans \cite{durbin2018some}}
\label{fig:lesvsransbackstep}
\end{center}
\end{figure}
%%\vspace{-2em}










%%% RECTANCULAR CYLINDER DNS
%%\vspace{-2em}
% \begin{figure}[htb]
\begin{figure}[H]
\begin{center}
\includegraphics[width=0.45\textwidth]{Images/logan/cimarelli2018direct_vortices.pdf}
\caption{ DNS square cylinder vortex locations Re=3000 \cite{cimarelli2018direct} }
\label{fig:dnsRectCylVortices}
\end{center}
\end{figure}
%%\vspace{-2em}


Fig. 4. Streamlines of the mean velocity field (U,V) (x,y)  The green lines show the primary vortex, the red lines mark the secondary vortex and the black lines denote the wake vortex. The red dots denote the locations of the probes used for the computation of time spectra in section x5. (For interpretation of the references to color in this figure legend, the reader is referred to the Web version of this article.)

%%\vspace{-2em}
% \begin{figure}[htb]
\begin{figure}[H]
\begin{center}
\includegraphics[width=0.45\textwidth]{Images/logan/cimarelli2018direct_pressure.pdf}
\caption{ DNS square cylinder mean pressure distribution Re=3000 \cite{cimarelli2018direct} }
\label{fig:dnsRectCylPressure}
\end{center}
\end{figure}
%%\vspace{-2em}

Fig. 5. Isocontours of the mean pressure field P(x,y). The dashed lines report the location of the primary vortex, secondary vortex and wake vortex.


%%\vspace{-2em}
% \begin{figure}[htb]
\begin{figure}[H]
\begin{center}
\includegraphics[width=0.5\textwidth]{Images/logan/cimarelli2018direct_vorticity.pdf}
\caption{ DNS square cylinder vorticity contours Re=3000 \cite{cimarelli2018direct} }
\label{fig:dnsRectCylPressure}
\end{center}
\end{figure}
%%\vspace{-2em}

Fig. 10. Instantaneous isosurfaces of $\lambda_2=-2$ colored with $y$. Perspective, top and lateral views in (a), (b) and (c) plots, respectively.































%%%%%%%%%%%%%%%%%%%%%%%%%%%%%%%%%%%%%%%%%%%%%%%%%%%%%%%%%%%%%%%%%%%%%%%%
\section{Current State of Bluff-Body Turbulence Analysis} \label{sec:currentstate}
%%%%%%%%%%%%%%%%%%%%%%%%%%%%%%%%%%%%%%%%%%%%%%%%%%%%%%%%%%%%%%%%%%%%%%%%

\begin{itemize}
    \item Current State of Knowledge
    \item Remaining Challenges
\end{itemize}


%%%%%%%%%%%%%%%%%%%%%%%%%%%%%%%%%%%%%%%%%%%%%%%%%%%%%%%%%%%%%%%%%%%%%%%%
\subsection{Experimental Methods} \label{subsec:currentstateexperimental}

\textcolor{red}{\emph{FZ}}


%%%%%%%%%%%%%%%%%%%%%%%%%%%%%%%%%%%%%%%%%%%%%%%%%%%%%%%%%%%%%%%%%%%%%%%%
\subsection{Computational Methods} \label{subsec:currentstatecomputational}



\textcolor{red}{\emph{LH}}














%%%%%%%%%%%%%%%%%%%%%%%%%%%%%%%%%%%%%%%%%%%%%%%%%%%%%%%%%%%%%%%%%%%%%%%%
\section{Conclusions}
%%%%%%%%%%%%%%%%%%%%%%%%%%%%%%%%%%%%%%%%%%%%%%%%%%%%%%%%%%%%%%%%%%%%%%%%

\textcolor{red}{\emph{LH\&FZ}}



%%%%%%%%%%%%%%%%%%%%%%%%%%%%%%%%%%%%%%%%%%%%%%%%%%%%%%%%%%%%%%%%%%%%%%%%
\section*{Acknowledgments} %%%%%%%%%%%%%%%%%%%%%%%%%%%%%%%%%%%%%%%%%%%%%
%%%%%%%%%%%%%%%%%%%%%%%%%%%%%%%%%%%%%%%%%%%%%%%%%%%%%%%%%%%%%%%%%%%%%%%%

\textcolor{red}{\emph{LH\&FZ}}

%%%%%%%%%%%%%%%%%%%%%%%%%%%%%%%%%%%%%%%%%%%%%%%%%%%%%%%%%%%%%%%%%%%%%%%%
%%% BIBLIOGRAPHY %%%%%%%%%%%%%%%%%%%%%%%%%%%%%%%%%%%%%%%%%%%%%%%%%%%%%%%
%%%%%%%%%%%%%%%%%%%%%%%%%%%%%%%%%%%%%%%%%%%%%%%%%%%%%%%%%%%%%%%%%%%%%%%%

%SAMPLE CITATIONS TO SEE CITATION FORMATTING
\textcolor{red}{Example citations}

\cite{nakamura1993bluffbody}

%bibliography from .bib file, filename goes in {}
%NOTE: References must be cited with "\cite" command to appear in bibliography
%Types of Refs: article, book, conference=inproceedings, manual, mastersthesis, phdthesis, techreport, unpublished, misc (see "new-aiaa.bst")
%when you first initialize .bib file, might need to have plain text in front of \cite command to get sublime text to recognize bibliography file
\bibliography{BluffBodyTurb}

\end{document}
